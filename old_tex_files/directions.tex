\documentclass[12pt]{article}

\usepackage{setspace}
\usepackage[thmmarks]{ntheorem}
\usepackage{amssymb, amsfonts, amsmath, mathrsfs, color, fancyhdr, tikz-cd, adjustbox, bbm, xcolor, wasysym}
\usepackage[ntheorem,framemethod=TikZ]{mdframed}
\usepackage{diagbox}
%Disabling for now to speed up compilation
\usepackage{hyperref}
\hypersetup{
	colorlinks = true,
	linkcolor = [rgb]{0,0,0.5},
	citecolor = [rgb]{0.6,0,0},
	urlcolor = [rgb]{0,0,0.5}
}

% Suppress mdframed telling us about "bad breaks". I can already see them.
\usepackage{silence}
\WarningFilter{mdframed}{You got a bad break}
\makeatletter
\mdf@PackageWarning{You got a bad break\MessageBreak
  because the last split box is empty\MessageBreak
  You have to change the settings}
\makeatother

\usepackage[style=alphabetic, bibencoding=utf8]{biblatex}
%Set the bibliography file
\bibliography{sources}

%Document-Specific includes
\usepackage{ytableau}

%Replacement for the old geometry package
\usepackage{fullpage}

%Input my definitions
\input{mydefs.tex}

%%%%%%%%%%%%%%%%%%%%%%% Customize Below %%%%%%%%%%%%%%%%%%%%%%%%%%%%%%
%%%%%%%%%%%%%%%%%%%%%%%%%%%%%%%%%%%%%%%%%%%%%%%%%%%%%%%%%%%%%%%%%%%%%%

%header stuff
\setlength{\headsep}{24pt}  % space between header and text
\pagestyle{fancy}     % set pagestyle for document
\lhead{Research Discussion Notes} % put text in header (left side)
\rhead{Nico Courts} % put text in header (right side)
\cfoot{\itshape p. \thepage}
\setlength{\headheight}{15pt}
%\allowdisplaybreaks

% Document-Specific Macros
\DeclareMathOperator{\Spc}{Spc}
\DeclareMathOperator{\Pol}{Pol}
\newcommand{\vi}{\mathbf{i}}
\newcommand{\vj}{\mathbf{j}}
\newcommand{\vk}{\mathbf{k}}
\newcommand{\vl}{\mathbf{l}}
\newcommand{\vm}{\mathbf{m}}
\DeclareMathOperator{\rad}{rad}
\DeclareMathOperator{\ev}{ev}
\DeclareMathOperator{\add}{\mathbf{add}}
\DeclareMathOperator{\supph}{supp^\mathit{hyp}}

\begin{document}
%make the title page
\title{Research directions}
\author{Nico Courts}
\date{}
\maketitle

\section{Overview}
The main problem that we are looking to solve it to establish whether the tensor product identity holds for bosonized quantum complete intersections $\Lambda=k_q[x_1,\dots,x_n]/(x_1^p,\dots, x_n^p)\rtimes G$ for a(n elementary abelian) group $G$: 
\[\supp M\cap \supp N = \supp M\otimes N,\quad\forall M,N\in \lmod{\Lambda}\]

\begin{itemize}
    \item To answer in the \textbf{negative,} we will want to find a pair of modules for which this property fails. Likely we will need to use $n\ge 3$. We may need to consider arbitrary $q$-commutativity relations.
    \item To answer in the \textbf{positive,} we'll need to do more theoretical work to show that the property is always satisfied.
\end{itemize}

\subsection{Support}
Of course underlying the whole problem is \textit{support}, which at its simplest (when $\Lambda$ is commutative) is $\supp M=\{\frakp\in\Spec\Lambda:M_\frakp =0\}$. In our case, we have to define this in terms of the cohomology ring $H^\bullet(\Lambda, k)\eqdef\Ext^\bullet_A(k,k)$ which is defined for any Hopf algebra. In our case, we also have other varieties at our disposal: the hypersurface support variety (determined by degenerate points where the projective dimension is finite) and $\pi$-support variety (defined using the fact that $\Lambda$ is analogous to an elementary abelian group). These notions should all agree.

A problem that will arise here is that we will be doing our computations on the truncated quantum polynomial algebra although we are actually interested in the support of the \textit{bosonized} version. We will have to do a little work to show that the support of the two coincide.

\subsection{Matrix factorizations}
We have already established that there exist matrix factorizations for any choice of parameters and any dimension given by an iterative construction. What remains to make these useful to us is to show that we can use some notion of these matrices to identify points in the hypersurface support variety. 

The idea (analogous to the work of Avramov \& Iyengar) is that the points will correspond to those where the rank of the matrix $(\begin{smallmatrix}0 & A\\B&0\end{smallmatrix})$ dips below half. I need to put together an exact statement and then prove that this is the case. We will call this the \textbf{rank variety} (which makes sense but collides slightly with Carlson's notation, which we are calling the $\pi$-support variety).

\section{Side quests}
\begin{itemize}
    \item 
\end{itemize}

\end{document}